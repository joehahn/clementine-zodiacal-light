\documentclass[12pt]{letter}

\addtolength{\textheight}{5cm}
\addtolength{\topmargin}{-5cm}
\addtolength{\oddsidemargin}{-1cm}
\addtolength{\textwidth}{2cm}

\begin{document}
\parindent=25pt

\address{Lunar and Planetary Institute\\
         3600 Bay Area Boulevard\\
         Houston, TX 77058\\
	 hahn@lpi.usra.edu\\
	 281--486--2113}

\signature{Joseph M. Hahn}
\date{\today}

\begin{letter}{
Mrs.\ Bernice Zook\\
1703 Bowline\\
Houston, TX 77062
}
\opening{\noindent Dear Mrs.\ Zook,}

\indent I have enclosed a few copies of the paper I recently
published with your late husband Herb. This paper describes a
project I worked on with Herb over the past two years until he
passed away. It describes an analysis of some spacecraft
observations of the interplanetary dust complex which, as you
know, was one of Herb's scientific passions. I am quite proud of
this paper since it manages to tackle the difficult question of
whether interplanetary dust is coming predominantly from
asteroids or comets. This paper has already generated
considerable interest throughout the interplanetary
dust community, and I am also quite
sure he would have been very pleased to see our work
appear on the cover of {\it Icarus}, which is our flagship
journal for planetary science.

Herb was instrumental to this project at all levels. It was his
idea to use the navigation cameras onboard the Clementine
spacecraft (whose primary mission was to study the Moon rather
than interplanetary dust) to observe this dust, and it was
also his persistent lobbying of the Clementine team that
eventually made the experiment a reality. I also learned a lot
about interplanetary dust from Herb. He was a true gentleman of
science, and he was alway in a remarkably upbeat mood despite
his health problems.
He is missed by myself and his many colleagues.

\closing{With kind regards,}
\end{letter}
\end{document}
