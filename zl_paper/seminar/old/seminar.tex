%
%	latex seminar.tex
%
%       to dvips in landscape mode for printing on paper:
%       dvips -t landscape -o seminar.ps seminar -Pcmz -Pamz
%
%	to dvips for slides & pdf slideshow:
%	dvips -o seminar.ps seminar -T 11.0in,7.33in -Pcmz -Pamz
%
\documentclass[20pt,35mmSlide,landscape]{foils}
\usepackage{color,times,psfig}
\setlength{\textwidth}{10in}
\setlength{\oddsidemargin}{-0.5in}
%\setlength{\evensidemargin}{-0.5in}

\newcommand{\gtrsim}{\mbox{ \lower0.5ex\hbox{$\sim$} \kern-1.1em 
                 \raise0.5ex\hbox{$>$}}}
\newcommand{\lesssim}{\mbox{ \lower0.5ex\hbox{$\sim$} \kern-1.1em 
                 \raise0.5ex\hbox{$<$}}}

\definecolor{light-blue}{cmyk}{0.1, 0.0, 0.0, 0.0}
\definecolor{light-yellow}{cmyk}{0.0, 0.0, 0.1, 0.0}
\definecolor{light-red}{rgb}{1.0, 0.9, 0.9}
\definecolor{light-green}{rgb}{0.9, 1.0, 0.9}

%%%%%%%%%%%%%%%%%%%%%%%%%%%%%%%%%%%%%%%%%%%%%%%%%%%%%%%%%%%%%%%%%
% 1. Title page                                                 %
%%%%%%%%%%%%%%%%%%%%%%%%%%%%%%%%%%%%%%%%%%%%%%%%%%%%%%%%%%%%%%%%%

\pagecolor{light-green}

\title{\vspace*{-5ex}Clementine Observations of the Zodiacal
Light\\
and the Dust Content of the Inner Solar System\vspace*{1.5ex}}

\author{Joseph M.\ Hahn (LPI)\vspace*{4ex}\\
with Herb Zook (NASA/JSC), Bonnie Cooper (OSS),
and Sunny Sunkara (LPI)\vspace*{3ex}}

\date{\today}
\MyLogo{}

\begin{document}
\maketitle

%%%%%%%%%%%%%%%%%%%%%%%%%%%%%%%%%%%%%%%%%%%%%%%%%%%%%%%%%%%%%%%%%
% 2. Herb								%
%%%%%%%%%%%%%%%%%%%%%%%%%%%%%%%%%%%%%%%%%%%%%%%%%%%%%%%%%%%%%%%%%
\pagecolor{light-yellow}

\begin{figure}
\vspace*{-0.75in}\centerline{\psfig{figure=figs/zook.ps,width=10.0in,height=6.0in}}
\end{figure}

%%%%%%%%%%%%%%%%%%%%%%%%%%%%%%%%%%%%%%%%%%%%%%%%%%%%%%%%%%%%%%%%%
% 3. ZL pic								%
%%%%%%%%%%%%%%%%%%%%%%%%%%%%%%%%%%%%%%%%%%%%%%%%%%%%%%%%%%%%%%%%%
\pagecolor{light-yellow}

\parbox{10in}{
\parbox{5in}{
\begin{figure}[t]
\vspace*{-6ex}\psfig{figure=figs/zl_pic.ps,height=5.0in,width=4.6in}
\end{figure}
{\hspace{10ex}\footnotesize photo by Marco Fulle.}
}
\parbox[c]{4.5in}{

\vspace*{-4ex}The zodiacal light (ZL) is sunlight that is
scattered and/or reradiated by interplanetary dust.\\

The inner ZL is observed towards the sun,
usually at optical wavelengths.\\

The outer ZL is observed away from the sun,
usually at infrared wavelengths.
}
}

%%%%%%%%%%%%%%%%%%%%%%%%%%%%%%%%%%%%%%%%%%%%%%%%%%%%%%%%%%%%%%%%%
% 3. Why?								%
%%%%%%%%%%%%%%%%%%%%%%%%%%%%%%%%%%%%%%%%%%%%%%%%%%%%%%%%%%%%%%%%%
\pagecolor{light-yellow}

\begin{center}
\vspace*{-1in}{\Large Why Study Interplanetary
Dust?\vspace*{1ex}}
\end{center}

``Someone unfamiliar with astrophysical problems would certainly
consider the study of interplanetary dust as an exercise of pure
acedemic interest and may even smile at the fact that much
theoretical machinery is devoted to tiny dust grains''

{\small Philippe Lamy, 1975, Ph.D.\ thesis.}

%%%%%%%%%%%%%%%%%%%%%%%%%%%%%%%%%%%%%%%%%%%%%%%%%%%%%%%%%%%%%%%%%
% 4. Why again?								%
%%%%%%%%%%%%%%%%%%%%%%%%%%%%%%%%%%%%%%%%%%%%%%%%%%%%%%%%%%%%%%%%%
\newpage
\pagecolor{light-yellow}
\small

\vspace*{-1.5in}\begin{itemize}

\item Dust are samples of small bodies that formed in remote
niches througout the solar system, and they place
constraints on conditions in the solar nebula during the
planet--forming epoch.\vspace*{1ex}

\begin{itemize}

\item dust from asteroids tell us of solar nebula conditions
at $r\sim3$ AU\vspace*{1ex}

\item dust from long--period Oort Cloud comets tell us of nebula
conditions at $5\lesssim r\lesssim30$ AU\vspace*{1ex}

\item dust from short--period Jupiter--Family comets tell us
of conditions in the\\ Kuiper Belt at $r\gtrsim30$ AU\\

\end{itemize}

\item IF the information carried by dust samples
(collected by U2 aircraft, Stardust, spacecraft impact
experiments, {\it etc.}) are indeed decipherable, then their
minerology will inform us of nebula conditions and its history
over $3\lesssim r\lesssim30$ AU.\\

\item However interpreting this dust requires understanding their
{\sl sources} (asteroid \& comets), their {\sl spatial
distributions}, transport mechanisms, and sampling biases
(e.g., certain sources may be more effective at delivering dust
to your detector than other sources).

\end{itemize}

%%%%%%%%%%%%%%%%%%%%%%%%%%%%%%%%%%%%%%%%%%%%%%%%%%%%%%%%%%%%%%%%%
% 5. History								%
%%%%%%%%%%%%%%%%%%%%%%%%%%%%%%%%%%%%%%%%%%%%%%%%%%%%%%%%%%%%%%%%%
\pagecolor{light-yellow}
\normalsize

\parbox{10in}{
\parbox{6.5in}{
\begin{figure}[l]
\vspace*{-1in}\psfig{figure=figs/fraction.ps,height=6.0in,width=6.0in}
\end{figure}
}
\parbox[c]{3.5in}{

\begin{center}
{\large The Abundance of Asteroidal \& Cometary Dust
Grains:\vspace*{2ex}}
\end{center}
an oversimplified and incomplete history of interplanetary
dust studies.\vspace*{1in}

}
}

%%%%%%%%%%%%%%%%%%%%%%%%%%%%%%%%%%%%%%%%%%%%%%%%%%%%%%%%%%%%%%%%%
% 6. Dust Bands								%
%%%%%%%%%%%%%%%%%%%%%%%%%%%%%%%%%%%%%%%%%%%%%%%%%%%%%%%%%%%%%%%%%
\newpage
\pagecolor{light-yellow}
\normalsize

\vspace*{-1.5in}\begin{center}
{\small Dust Bands in the Outer Zodiacal Light}
\end{center}
\begin{figure}[l]
\vspace*{-0.18in}\psfig{figure=figs/IRAS_sykes.ps,height=4.4in,width=10.0in}
\end{figure}
\vspace*{-2.5ex}{\small
Dust bands are due to collisions among Themis--family asteroids
($i=1.4^\circ$), Koronis family ($i=2.1^\circ$), and
Eos ($i=9.4^\circ$) (Dermott {\it et al.} 1984).\\
Note also the dust trails from comets Encke \& Tempel 2
}

%%%%%%%%%%%%%%%%%%%%%%%%%%%%%%%%%%%%%%%%%%%%%%%%%%%%%%%%%%%%%%%%%
% 7. Dust Bands	2							%
%%%%%%%%%%%%%%%%%%%%%%%%%%%%%%%%%%%%%%%%%%%%%%%%%%%%%%%%%%%%%%%%%
\newpage
\pagecolor{light-yellow}
\small

\vspace*{-12ex}\begin{center}
{\small How Asteroid Families Produce Dust Band Pairs}
\end{center}
\vspace*{-2ex}\parbox{10in}{
\parbox{8.5in}{
\begin{figure}[t]
\psfig{figure=figs/dust_bands.ps,height=2.5in,width=8.0in}
\end{figure}
}
\hfil\parbox[c]{1.5in}{figure from Kortenkamp {\it et al.} (2001)
}
}
\parbox{10in}{\small
Asteroids in a family are fragments of a larger parent body
having common $(a,e,i)$, which leads to frequent collisions
among members.\vspace*{1ex}

Collisions generate dust which spirals sunwards due to
Poynting--Robertson drag, producing a sheet of dust having an
angular thickness $\sim2i$.\vspace*{0.8ex}

The grain's vertical motion is oscillatory, so they are densest
at the higher latitudes where their  vertical velocity is
lowest $\Rightarrow$ pair of dust bands at latitude
\vspace*{1ex}$\pm i$.\vspace*{0.7ex}

This dust sheet is also warped by the giant planets'
secular gravitational perturbations.
}

%%%%%%%%%%%%%%%%%%%%%%%%%%%%%%%%%%%%%%%%%%%%%%%%%%%%%%%%%%%%%%%%%
% 8. Clementine								%
%%%%%%%%%%%%%%%%%%%%%%%%%%%%%%%%%%%%%%%%%%%%%%%%%%%%%%%%%%%%%%%%%
\newpage
\pagecolor{light-yellow}
\normalsize

\parbox{10in}{
\parbox{5.5in}{
\begin{figure}[t]
\vspace*{-6ex}\psfig{figure=figs/Clementine.ps,height=5.0in,width=5.0in}
\end{figure}
}
\hfil\parbox[c]{4.0in}{
\begin{center}
{\large Clementine Observations of the Inner Zodiacal
Light\vspace*{0.6in}}
\end{center}

``We don't need a lot of fancy Ph.D's to build a
spacecraft.''\vspace*{2ex}\\
{\small Lt. Col. Pedro Rustan,\\
Clementine program manager}\vspace*{1in}
}
}

%%%%%%%%%%%%%%%%%%%%%%%%%%%%%%%%%%%%%%%%%%%%%%%%%%%%%%%%%%%%%%%%%
% 9. Star Tracker							%
%%%%%%%%%%%%%%%%%%%%%%%%%%%%%%%%%%%%%%%%%%%%%%%%%%%%%%%%%%%%%%%%%
\newpage
\pagecolor{light-yellow}
\normalsize

\parbox{10in}{
\parbox{5.5in}{
\begin{figure}[t]
\vspace*{-9ex}\psfig{figure=figs/pretty_picture.ps,height=3.3in,width=5.0in}
\end{figure}
}
\hfil\parbox[c]{4.0in}{
\begin{center}
{\large The Star Tracker Camera\vspace*{2in}}
\end{center}
}
}

\parbox{10in}{\small
Hundreds of optical images of the inner ZL were acquired by 
wide--angle star tracker cameras that are ordinarily used for
spacecraft navigation.\vspace*{1ex}

With the Moon occulting the Sun, the inner ZL was observed over
elongations of \mbox{$2^\circ<\epsilon<30^\circ$}, or
$10R_\odot<r<$ Venus.\vspace*{1ex}

Lots of instrumental issues: no shutter!, dark--current
subtraction, flatfielding, calibrating...
}

%%%%%%%%%%%%%%%%%%%%%%%%%%%%%%%%%%%%%%%%%%%%%%%%%%%%%%%%%%%%%%%%%
% 10. Mosaic							%
%%%%%%%%%%%%%%%%%%%%%%%%%%%%%%%%%%%%%%%%%%%%%%%%%%%%%%%%%%%%%%%%%
%\newpage
\pagecolor{light-yellow}
\normalsize

\parbox{10in}{
\hspace*{-4ex}\parbox{7.25in}{
\begin{figure}[t]
\vspace*{-11ex}\psfig{figure=figs/mosaic.ps,height=7.0in,width=7.0in}
\end{figure}
}
\hfil\parbox[c]{2.75in}{\small
Clementine observed\\
7 distinct fields\\

Observed several planets: V,S,M,M,S,Mer.\\

integrated $m_V=-8.5$\\

Note:\\
\mbox{$m_V(\mbox{full Moon})=-12.7$},
\mbox{$m_V(\mbox{Venus})\ge-4.6$}\vspace*{1in}
}
}

%%%%%%%%%%%%%%%%%%%%%%%%%%%%%%%%%%%%%%%%%%%%%%%%%%%%%%%%%%%%%%%%%
% 11. Light Scattering						%
%%%%%%%%%%%%%%%%%%%%%%%%%%%%%%%%%%%%%%%%%%%%%%%%%%%%%%%%%%%%%%%%%
\newpage
\pagecolor{light-yellow}
\small

\vspace*{-1.5in}\parbox{10in}{
\parbox[c]{5.5in}{
\begin{center}
{\normalsize
\vspace*{5ex}A Simple Model for Interplanetary Dust\vspace*{5ex}}
\end{center}
assume the density of dust cross section varies as
\begin{displaymath}
\sigma(r,\beta)=\sigma_1\left(\frac{r}{r_1}\right)^{-\nu}h(\beta)
\end{displaymath}
then the ZL surface brightness is \mbox{(Aller {\it et al} 1967):}
}
\hfil\parbox{4.5in}{
\begin{figure}[t]
\psfig{figure=figs/geometry.ps,height=4.5in,width=4.5in}\vspace*{-1in}
\end{figure}
}
}
\vspace*{-2.3in}\parbox{10in}{
\begin{displaymath}
Z(\theta,\phi)=\frac{a\sigma_1 r_1}{\sin^{\nu+1}\epsilon}
\left(\frac{\Omega_\odot}{\pi\mbox{ sr}}\right)B_\odot
\int_\epsilon^\pi\psi(\varphi)h(\beta(\varphi))
\sin^\nu(\varphi)d\varphi
\end{displaymath}
where $a$=dust albedo, $B_\odot$=mean solar brightness, and\\
$\psi(\phi)$=Hong's (1985) empirical phase law for dust.
}

%%%%%%%%%%%%%%%%%%%%%%%%%%%%%%%%%%%%%%%%%%%%%%%%%%%%%%%%%%%%%%%%%
% 12. Profiles							%
%%%%%%%%%%%%%%%%%%%%%%%%%%%%%%%%%%%%%%%%%%%%%%%%%%%%%%%%%%%%%%%%%
\newpage
\pagecolor{light-yellow}
\small

\vspace*{-1.5in}\parbox{10in}{
\parbox[c]{4.5in}{
\begin{center}
{\normalsize
\vspace*{3ex}Dust Variations in the Ecliptic\vspace*{3ex}}
\end{center}
Note that $h(\beta)=1$ in the ecliptic, so
\begin{displaymath}
Z(\epsilon)\rightarrow1.8\times10^{-5}
\frac{a\sigma_1 r_1}{\sin^{\nu+1}\epsilon}B_\odot
\end{displaymath}
$\Rightarrow a\sigma_1=7.8\times10^{-22}$ cm$^2$/cm$^3$\\
$\Rightarrow \nu=1.45\pm0.05$.\\

Adopting $\sigma_1=4.6\times10^{-21}$ cm$^2$/cm$^3$\\
(Gr\"{u}n {\it et al.} 1985),\\
$\Rightarrow a=0.17$ within a factor of 2?
}
\hfil\parbox{5.0in}{
\begin{figure}[t]
\psfig{figure=figs/profiles.ps,height=5.0in,width=5.0in}\vspace*{-0.25in}
\end{figure}
}
}
\parbox{10in}{Note also the North--South and East--West asymmetries.}

%%%%%%%%%%%%%%%%%%%%%%%%%%%%%%%%%%%%%%%%%%%%%%%%%%%%%%%%%%%%%%%%%
% 13. Asymmetries						%
%%%%%%%%%%%%%%%%%%%%%%%%%%%%%%%%%%%%%%%%%%%%%%%%%%%%%%%%%%%%%%%%%
\newpage
\pagecolor{light-yellow}
\small

\vspace*{-1.5in}\parbox{10in}{
\parbox[c]{4.75in}{
\vspace*{-0.1in}\begin{figure}[t]
\psfig{figure=figs/longitudes.ps,height=4.0in,width=4.0in}
\end{figure}
The N--S asymmetry is probably due to the $i=3^\circ$ tilt
of the dust midplane detected by Helios (Leinert {\it et al} 1980)).
}
\hfil\parbox{4.75in}{
\vspace*{-1in}\begin{figure}[t]
\psfig{figure=figs/dust_orbits.ps,height=4.0in,width=4.0in}\vspace*{-0.0in}
\end{figure}
The E--W asymmetry is probably due to `pericenter glow' (Dermott {\it et al}).
}
}

\parbox{10in}{Tilt and pericenter glow are due to the secular part
of the giant planets' gravitational perturbations.}

\end{document}
