%
%	latex seminar.tex
%
%       to dvips in landscape mode for printing on paper:
%       dvips -t landscape -o seminar.ps seminar -Pcmz -Pamz
%
%	to dvips for slides & pdf slideshow:
%	dvips -o seminar.ps seminar -T 11.0in,7.33in -Pcmz -Pamz
%
%	to make PDF file
%	rm seminar.pdf; distill seminar.ps
%

\documentclass[20pt,35mmSlide,landscape]{foils}
\usepackage{color,times,psfig}
\setlength{\textwidth}{10in}
\setlength{\oddsidemargin}{-0.5in}
%\setlength{\evensidemargin}{-0.5in}

\newcommand{\gtrsim}{\mbox{ \lower0.5ex\hbox{$\sim$} \kern-1.1em 
                 \raise0.5ex\hbox{$>$}}}
\newcommand{\lesssim}{\mbox{ \lower0.5ex\hbox{$\sim$} \kern-1.1em 
                 \raise0.5ex\hbox{$<$}}}

\definecolor{light-blue}{cmyk}{0.1, 0.0, 0.0, 0.0}
\definecolor{light-yellow}{cmyk}{0.0, 0.0, 0.1, 0.0}
\definecolor{light-red}{rgb}{1.0, 0.9, 0.9}
\definecolor{light-green}{rgb}{0.9, 1.0, 0.9}

%%%%%%%%%%%%%%%%%%%%%%%%%%%%%%%%%%%%%%%%%%%%%%%%%%%%%%%%%%%%%%%%%
% 1. Title page                                                 %
%%%%%%%%%%%%%%%%%%%%%%%%%%%%%%%%%%%%%%%%%%%%%%%%%%%%%%%%%%%%%%%%%

\pagecolor{light-green}

\title{\vspace*{-5ex}Clementine Observations of the Zodiacal
Light\\
and the Dust Content of the Inner Solar System\vspace*{1.5ex}}

\author{Joseph M.\ Hahn (LPI)\vspace*{4ex}\\
with Herb Zook (NASA/JSC), Bonnie Cooper (OSS),
and Sunny Sunkara (LPI)\vspace*{3ex}}

\date{\today}
\MyLogo{}

\begin{document}
\maketitle

%%%%%%%%%%%%%%%%%%%%%%%%%%%%%%%%%%%%%%%%%%%%%%%%%%%%%%%%%%%%%%%%%
% 2. Herb								%
%%%%%%%%%%%%%%%%%%%%%%%%%%%%%%%%%%%%%%%%%%%%%%%%%%%%%%%%%%%%%%%%%
\pagecolor{light-yellow}

\begin{figure}
\vspace*{-0.75in}\centerline{\psfig{figure=figs/zook.ps,width=10.0in,height=6.0in}}
\end{figure}

%%%%%%%%%%%%%%%%%%%%%%%%%%%%%%%%%%%%%%%%%%%%%%%%%%%%%%%%%%%%%%%%%
% 3. ZL pic								%
%%%%%%%%%%%%%%%%%%%%%%%%%%%%%%%%%%%%%%%%%%%%%%%%%%%%%%%%%%%%%%%%%
\pagecolor{light-yellow}

\parbox{10in}{
\parbox{5in}{
\begin{figure}[t]
\vspace*{-6ex}\psfig{figure=figs/zl_pic.ps,height=5.0in,width=4.6in}
\end{figure}
{\hspace{10ex}\footnotesize photo by Marco Fulle.}
}
\parbox[c]{4.5in}{

\vspace*{-4ex}The zodiacal light (ZL) is sunlight that is
scattered and/or reradiated by interplanetary dust.\\

The inner ZL is observed towards the sun,
usually at optical wavelengths.\\

The outer ZL is observed away from the sun,
usually at infrared wavelengths.
}
}

%%%%%%%%%%%%%%%%%%%%%%%%%%%%%%%%%%%%%%%%%%%%%%%%%%%%%%%%%%%%%%%%%
% 3. Why?								%
%%%%%%%%%%%%%%%%%%%%%%%%%%%%%%%%%%%%%%%%%%%%%%%%%%%%%%%%%%%%%%%%%
\pagecolor{light-yellow}

\begin{center}
\vspace*{-1in}{\Large Why Study Interplanetary
Dust?\vspace*{1ex}}
\end{center}

``Someone unfamiliar with astrophysical problems would certainly
consider the study of interplanetary dust as an exercise of pure
academic interest and may even smile at the fact that much
theoretical machinery is devoted to tiny dust grains''

{\small Philippe Lamy, 1975, Ph.D.\ thesis.}

%%%%%%%%%%%%%%%%%%%%%%%%%%%%%%%%%%%%%%%%%%%%%%%%%%%%%%%%%%%%%%%%%
% 4. Why again?								%
%%%%%%%%%%%%%%%%%%%%%%%%%%%%%%%%%%%%%%%%%%%%%%%%%%%%%%%%%%%%%%%%%
\newpage
\pagecolor{light-yellow}
\small

\vspace*{-1.5in}\begin{itemize}

\item Dust are samples of small bodies that formed in remote
niches throughout the solar system, and they place
constraints on conditions in the solar nebula during the
planet--forming epoch.\vspace*{1ex}

\begin{itemize}

\item dust from asteroids tell us of solar nebula conditions
at $r\sim3$ AU\vspace*{1ex}

\item dust from long--period Oort Cloud comets tell us of nebula
conditions at $5\lesssim r\lesssim30$ AU\vspace*{1ex}

\item dust from short--period Jupiter--Family comets tell us
of conditions in the\\ Kuiper Belt at $r\gtrsim30$ AU\\

\end{itemize}

\item IF the information carried by dust samples
(collected by U2 aircraft, Stardust, spacecraft impact
experiments, {\it etc.}) are indeed decipherable, then their
mineralogy will inform us of nebula conditions and its history
over $3\lesssim r\lesssim30$ AU.\\

\item However interpreting this dust requires understanding their
{\sl sources} (asteroid \& comets), their {\sl spatial
distributions}, transport mechanisms, and sampling biases
(e.g., certain sources may be more effective at delivering dust
to your detector than other sources).

\end{itemize}

%%%%%%%%%%%%%%%%%%%%%%%%%%%%%%%%%%%%%%%%%%%%%%%%%%%%%%%%%%%%%%%%%
% 5. History								%
%%%%%%%%%%%%%%%%%%%%%%%%%%%%%%%%%%%%%%%%%%%%%%%%%%%%%%%%%%%%%%%%%
\pagecolor{light-yellow}
\normalsize

\parbox{10in}{
\parbox{6.5in}{
\begin{figure}[l]
\vspace*{-1in}\psfig{figure=figs/fraction.ps,height=6.0in,width=6.0in}
\end{figure}
}
\parbox[c]{3.5in}{

\begin{center}
{\large The Perceived Abundance of Asteroidal \& Cometary Dust
Grains vs.\ Time:\vspace*{2ex}}
\end{center}
an oversimplified and incomplete history of interplanetary
dust studies.\vspace*{1in}

}
}

%%%%%%%%%%%%%%%%%%%%%%%%%%%%%%%%%%%%%%%%%%%%%%%%%%%%%%%%%%%%%%%%%
% 6. Dust Bands								%
%%%%%%%%%%%%%%%%%%%%%%%%%%%%%%%%%%%%%%%%%%%%%%%%%%%%%%%%%%%%%%%%%
\newpage
\pagecolor{light-yellow}
\normalsize

\vspace*{-1.5in}\begin{center}
{\small Dust Bands in the Outer Zodiacal Light}
\end{center}
\begin{figure}[l]
\vspace*{-0.18in}\psfig{figure=figs/IRAS_sykes.ps,height=4.4in,width=10.0in}
\end{figure}
\vspace*{-2.5ex}{\small
Dust bands are due to collisions among Themis--family asteroids
($i=1.4^\circ$), Koronis family ($i=2.1^\circ$), and
Eos ($i=9.4^\circ$) (Dermott {\it et al.} 1984).\\
Note also the dust trails from comets Encke \& Tempel 2.
}

%%%%%%%%%%%%%%%%%%%%%%%%%%%%%%%%%%%%%%%%%%%%%%%%%%%%%%%%%%%%%%%%%
% 7. Dust Bands	2							%
%%%%%%%%%%%%%%%%%%%%%%%%%%%%%%%%%%%%%%%%%%%%%%%%%%%%%%%%%%%%%%%%%
\newpage
\pagecolor{light-yellow}
\small

\vspace*{-12ex}\begin{center}
{\small How Asteroid Families Produce Dust Band Pairs}
\end{center}
\vspace*{-2ex}\parbox{10in}{
\parbox{8.5in}{
\begin{figure}[t]
\psfig{figure=figs/dust_bands.ps,height=2.5in,width=8.0in}
\end{figure}
}
\hfil\parbox[c]{1.5in}{figure from Kortenkamp {\it et al.} (2001)
}
}
\parbox{10in}{\small
Asteroids in a family are fragments of a larger parent body
having common $(a,e,i)$, which leads to frequent collisions
among members.\vspace*{1ex}

Collisions generate dust which spirals sunwards due to
Poynting--Robertson drag, producing a sheet of dust having an
angular thickness $\sim2i$.\vspace*{0.8ex}

The grain's vertical motion is oscillatory, so they are densest
at the higher latitudes where their  vertical velocity is
lowest $\Rightarrow$ pair of dust bands at latitude
\vspace*{1ex}$\pm i$.\vspace*{0.7ex}

This dust sheet is also warped by the giant planets'
secular gravitational perturbations.
}

%%%%%%%%%%%%%%%%%%%%%%%%%%%%%%%%%%%%%%%%%%%%%%%%%%%%%%%%%%%%%%%%%
% 8. Clementine								%
%%%%%%%%%%%%%%%%%%%%%%%%%%%%%%%%%%%%%%%%%%%%%%%%%%%%%%%%%%%%%%%%%
\newpage
\pagecolor{light-yellow}
\normalsize

\parbox{10in}{
\parbox{5.5in}{
\begin{figure}[t]
\vspace*{-6ex}\psfig{figure=figs/Clementine.ps,height=5.0in,width=5.5in}
\end{figure}
}
\hfil\parbox[c]{4.0in}{
\begin{center}
{\large Clementine Observations of the Inner Zodiacal
Light\vspace*{0.6in}}
\end{center}

``We don't need a lot of fancy Ph.D's to build a
spacecraft.''\vspace*{2ex}\\
{\small Lt. Col. Pedro Rustan,\\
Clementine program manager}\vspace*{1in}
}
}

%%%%%%%%%%%%%%%%%%%%%%%%%%%%%%%%%%%%%%%%%%%%%%%%%%%%%%%%%%%%%%%%%
% 9. Star Tracker							%
%%%%%%%%%%%%%%%%%%%%%%%%%%%%%%%%%%%%%%%%%%%%%%%%%%%%%%%%%%%%%%%%%
\newpage
\pagecolor{light-yellow}
\normalsize

\parbox{10in}{
\parbox{5.5in}{
\begin{figure}[t]
\vspace*{-9ex}\psfig{figure=figs/pretty_picture.ps,height=3.5in,width=5.25in}
\end{figure}
}
\hfil\parbox[c]{4.0in}{
\begin{center}
{\large The Star Tracker Camera\vspace*{2in}}
\end{center}
}
}

\parbox{10in}{\small
Hundreds of optical images of the inner ZL were acquired by 
wide--angle star tracker cameras that are ordinarily used for
spacecraft navigation.\vspace*{1ex}

With the Moon occulting the Sun, the inner ZL was observed over
elongations of \mbox{$2^\circ<\epsilon<30^\circ$}, or
$10R_\odot<r<$ Venus.\vspace*{1ex}

Lots of instrumental issues: no shutter!, dark--current
subtraction, flatfielding, calibrating...
}

%%%%%%%%%%%%%%%%%%%%%%%%%%%%%%%%%%%%%%%%%%%%%%%%%%%%%%%%%%%%%%%%%
% 10. Mosaic							%
%%%%%%%%%%%%%%%%%%%%%%%%%%%%%%%%%%%%%%%%%%%%%%%%%%%%%%%%%%%%%%%%%
%\newpage
\pagecolor{light-yellow}
\normalsize

\parbox{10in}{
\hspace*{-4ex}\parbox{7.25in}{
\begin{figure}[t]
\vspace*{-11ex}\psfig{figure=figs/mosaic.ps,height=7.0in,width=7.0in}
\end{figure}
}
\hfil\parbox[c]{2.75in}{\small
Clementine observed\\
7 distinct fields\\

Observed several planets: V,S,M,M,S,Mer.\\

integrated $m_V=-8.5$\\

Note:\\
\mbox{$m_V(\mbox{full Moon})=-12.7$},
\mbox{$m_V(\mbox{Venus})\ge-4.6$}\vspace*{1in}
}
}

%%%%%%%%%%%%%%%%%%%%%%%%%%%%%%%%%%%%%%%%%%%%%%%%%%%%%%%%%%%%%%%%%
% 11. Light Scattering						%
%%%%%%%%%%%%%%%%%%%%%%%%%%%%%%%%%%%%%%%%%%%%%%%%%%%%%%%%%%%%%%%%%
\newpage
\pagecolor{light-yellow}
\small

\vspace*{-1.5in}\parbox{10in}{
\parbox[c]{5.5in}{
\begin{center}
{\normalsize
\vspace*{5ex}A Simple Model for Interplanetary Dust\vspace*{1ex}}
\end{center}
\noindent assume the density of dust cross section varies as
{\normalsize
\begin{displaymath}
\sigma(r,\beta)=\sigma_1\left(\frac{r}{r_1}\right)^{-\nu}h(\beta)
\end{displaymath}
}
\noindent then the ZL surface brightness is
\mbox{(Aller {\it et al} 1967):}
}
\hfil\parbox{4.5in}{
\begin{figure}[t]
\psfig{figure=figs/geometry.ps,height=4.5in,width=4.5in}\vspace*{-1in}
\end{figure}
}
}
\vspace*{-2.3in}\parbox{10in}{
{\normalsize
\begin{displaymath}
Z(\theta,\phi)=\frac{a\sigma_1 r_1}{\sin^{\nu+1}\epsilon}
\left(\frac{\Omega_\odot}{\pi\mbox{ sr}}\right)B_\odot
\int_\epsilon^\pi\psi(\varphi)h(\beta(\varphi))
\sin^\nu(\varphi)d\varphi
\end{displaymath}
}
where $a$=dust albedo, $B_\odot$=mean solar brightness, and\\
$\psi(\phi)$=Hong's (1985) empirical phase law for dust.
}

%%%%%%%%%%%%%%%%%%%%%%%%%%%%%%%%%%%%%%%%%%%%%%%%%%%%%%%%%%%%%%%%%
% 12. Profiles							%
%%%%%%%%%%%%%%%%%%%%%%%%%%%%%%%%%%%%%%%%%%%%%%%%%%%%%%%%%%%%%%%%%
\newpage
\pagecolor{light-yellow}
\normalsize

\vspace*{-2in}\parbox{10in}{
\parbox[c]{4.5in}{
\begin{center}
{\normalsize
\vspace*{3ex}Dust Variations in the Ecliptic\vspace*{3ex}}
\end{center}
Note that $h(\beta)=1$ in the ecliptic, so
{\normalsize
\begin{displaymath}
Z(\epsilon)\rightarrow1.8\times10^{-5}
\frac{a\sigma_1 r_1}{\sin^{\nu+1}\epsilon}B_\odot
\end{displaymath}
}
$\Rightarrow a\sigma_1=7.8\times10^{-22}$ cm$^2$/cm$^3$\\
$\Rightarrow \nu=1.45\pm0.05$.\\

Adopting $\sigma_1=4.6\times10^{-21}$ cm$^2$/cm$^3$\\
(Gr\"{u}n {\it et al.} 1985),\\
$\Rightarrow a=0.17$ within a factor of 2?
}
\hfil\parbox{5.0in}{
\begin{figure}[t]
\psfig{figure=figs/profiles.ps,height=5.0in,width=5.0in}\vspace*{-0.0in}
\end{figure}
}
}

\parbox{10in}{Note also the North--South and East--West asymmetries.}

%%%%%%%%%%%%%%%%%%%%%%%%%%%%%%%%%%%%%%%%%%%%%%%%%%%%%%%%%%%%%%%%%
% 13. Asymmetries						%
%%%%%%%%%%%%%%%%%%%%%%%%%%%%%%%%%%%%%%%%%%%%%%%%%%%%%%%%%%%%%%%%%
\newpage
\pagecolor{light-yellow}
\small

\vspace*{-1.5in}\parbox{10in}{
\parbox[c]{4.75in}{
\vspace*{-0.7in}\begin{figure}[t]
\psfig{figure=figs/longitudes.ps,height=4.5in,width=4.5in}
\end{figure}
\vspace*{-3ex}\parbox{4.5in}{The N--S asymmetry is likely due to the
$i=3^\circ$ tilt of the dust midplane detected by Helios
(Leinert {\it et al} 1980).}
}
\hfil\parbox{4.75in}{
\vspace*{-0.4in}\begin{figure}[t]
\psfig{figure=figs/dust_orbits.ps,height=4.0in,width=4.0in}\vspace*{0.2in}
\end{figure}
\parbox{4in}{The E--W asymmetry is probably due to `pericenter glow'
(Dermott {\it et al}).\vspace*{2.5ex}}
}
}
\parbox{10in}{Tilt and pericenter glow are due to the secular part
of the giant planets' gravitational perturbations.}

%%%%%%%%%%%%%%%%%%%%%%%%%%%%%%%%%%%%%%%%%%%%%%%%%%%%%%%%%%%%%%%%%
% 14. Vertical Variations					%
%%%%%%%%%%%%%%%%%%%%%%%%%%%%%%%%%%%%%%%%%%%%%%%%%%%%%%%%%%%%%%%%%
\newpage
\pagecolor{light-yellow}
\normalsize

\vspace*{-1in}\begin{center}
{\large The Dust Vertical Variations}
\end{center}
\noindent Recall that the dust surface brightness $Z$ varies as
\begin{displaymath}
Z\propto\int_\epsilon^\pi\psi(\varphi)h(\beta(\varphi))
\sin^\nu(\varphi)d\varphi
\end{displaymath}
where the latitude distribution $h(\beta)$ depends upon
the dust inclination distribution $g(i)$:
\begin{displaymath}
h(\beta)=\int^{\pi/2}_\beta
\frac{g(i)di}{\sqrt{\sin^2i-\sin^2\beta}}.
\end{displaymath}

\noindent Note: an isotropic dust shell has $g(i)\propto\sin(i)$.

%%%%%%%%%%%%%%%%%%%%%%%%%%%%%%%%%%%%%%%%%%%%%%%%%%%%%%%%%%%%%%%%%
% 15. Vertical Variations 2					%
%%%%%%%%%%%%%%%%%%%%%%%%%%%%%%%%%%%%%%%%%%%%%%%%%%%%%%%%%%%%%%%%%
\newpage
\pagecolor{light-yellow}
\normalsize

\vspace*{-1in}\begin{center}
{\large Inferring the Dust Vertical Distribution}
\end{center}

\begin{itemize}

\item specify a trial inclination distribution $g(i)$

\begin{itemize}

\item assume the dust have $g(i)$ similar to their sources
     (e.g., asteroids \& comets).

\end{itemize}

\item compute latitude distribution $h(\beta)$ and
      the surface brightness $Z_{model}$

\item compare to $Z_{observed}$

\end{itemize}
%%%%%%%%%%%%%%%%%%%%%%%%%%%%%%%%%%%%%%%%%%%%%%%%%%%%%%%%%%%%%%%%%
% 16. Inclination Distributions					%
%%%%%%%%%%%%%%%%%%%%%%%%%%%%%%%%%%%%%%%%%%%%%%%%%%%%%%%%%%%%%%%%%
\nopagebreak
\pagecolor{light-yellow}
\small
\vspace*{-2in}\hspace*{-0.7in}\parbox{11in}{
\parbox{5.5in}{
\begin{figure}[t]
\psfig{figure=figs/asteroids.ps,height=3.3in,width=5.5in}\vspace*{-1ex}
\end{figure}
\begin{center}
{\normalsize ansatz: $g_j(i)\propto\sin(i)
\exp[-(i/\sigma_j)^2/2]$\vspace*{-3ex}}
\end{center}
\begin{figure}
\psfig{figure=figs/JFCs.ps,height=3.3in,width=5.5in}
\end{figure}
}
\parbox{5.5in}{
\begin{figure}[t]
\psfig{figure=figs/HTCs.ps,height=3.3in,width=5.5in}
\end{figure}
\begin{center}
{\tiny \ \vspace*{-3ex}}
\end{center}
\begin{figure}
\psfig{figure=figs/OCCs.ps,height=3.3in,width=5.5in}
\end{figure}
}
}

%%%%%%%%%%%%%%%%%%%%%%%%%%%%%%%%%%%%%%%%%%%%%%%%%%%%%%%%%%%%%%%%%
% 17. models							%
%%%%%%%%%%%%%%%%%%%%%%%%%%%%%%%%%%%%%%%%%%%%%%%%%%%%%%%%%%%%%%%%%
\newpage
\pagecolor{light-yellow}
\normalsize

\vspace*{-1.5in}\parbox{10in}{
{\large Evidently there are 3 source populations having distinct\\
inclination distributions:}

\begin{itemize}

\item low--$i$ population (asteroids + JFCs) with
$\sigma_{low}\simeq7^\circ$

\item high--$i$ population (HTCs) with
$\sigma_{high}\simeq33^\circ$

\item isotropic population (OCCs) with $g(i)\propto\sin i$

\end{itemize}
}
\vspace*{-1.5in}\hspace*{-0.3in}\parbox{11in}{
\parbox[t]{3.67in}{
\begin{figure}
\psfig{figure=figs/low.ps,height=3.5in,width=3.5in}
\end{figure}
}
\parbox[t]{3.66in}{
\begin{figure}
\psfig{figure=figs/hi.ps,height=3.5in,width=3.5in}
\end{figure}
}
\parbox[t]{3.67in}{
\begin{figure}
\psfig{figure=figs/iso.ps,height=3.5in,width=3.5in}
\end{figure}
}
}

%%%%%%%%%%%%%%%%%%%%%%%%%%%%%%%%%%%%%%%%%%%%%%%%%%%%%%%%%%%%%%%%%
% 18. Fit							%
%%%%%%%%%%%%%%%%%%%%%%%%%%%%%%%%%%%%%%%%%%%%%%%%%%%%%%%%%%%%%%%%%
\newpage
\pagecolor{light-yellow}
\normalsize

\vspace*{-2in}\parbox{10in}{
\parbox[c]{3.5in}{
\begin{center}
{\large
\vspace*{3ex}Fitting the Observations\vspace*{2ex}}
\end{center}
Parameters for the best fit:
\begin{itemize}
\item low--$i$: $\nu_{low}\simeq1.0$ and\\
$f_{low}=0.45\pm0.13$.\vspace*{0ex}
\item high--$i$: $\nu_{high}\simeq1.45$ and\\
$f_{high}=0.50\pm0.02$.
\item isotropic: $\nu_{iso}\simeq2.0$ and\\
 $f_{iso}=0.05\pm0.02$.
\item $a\sigma_1r_1=(1.2\pm0.1)\times10^{-8}$
\end{itemize}

}
\hfil\parbox{6.5in}{
\begin{figure}[t]
\psfig{figure=figs/fit_nu_ast=1.0_nu_occ=2.0.ps,height=6.0in,width=6.0in}\vspace*{-0.25in}
\end{figure}
}
}

%%%%%%%%%%%%%%%%%%%%%%%%%%%%%%%%%%%%%%%%%%%%%%%%%%%%%%%%%%%%%%%%%
% 19. Latitude & Inclination Distributions			%
%%%%%%%%%%%%%%%%%%%%%%%%%%%%%%%%%%%%%%%%%%%%%%%%%%%%%%%%%%%%%%%%%
\newpage
\pagecolor{light-yellow}
\normalsize


\vspace*{-1.6in}\begin{center}
The Inferred Inclination \& Latitude Distributions
\end{center}
\vspace*{-0.6in}\parbox{10in}{
\parbox{5in}{
\begin{figure}[t]
\psfig{figure=figs/inclination_dist.ps,height=4.65in,width=4.75in}
\end{figure}
}
\hfil\parbox{5in}{
\begin{figure}[t]
\psfig{figure=figs/latitude.ps,height=4.65in,width=4.75in}
\end{figure}
}

}
\vspace*{-1.3ex}\parbox{10in}{\small
Since $f_{low}=45\%\Rightarrow$ at most 45\%
of the ecliptic dust cross section is contributed by asteroids
(plus and unknown contribution by JFCs).\vspace*{0.75ex}\\
But at latitudes $\beta>15^\circ$, 90\% of the dust
is from HTCs and OCCs
}

%%%%%%%%%%%%%%%%%%%%%%%%%%%%%%%%%%%%%%%%%%%%%%%%%%%%%%%%%%%%%%%%%
% 20. I-pumping							%
%%%%%%%%%%%%%%%%%%%%%%%%%%%%%%%%%%%%%%%%%%%%%%%%%%%%%%%%%%%%%%%%%
\newpage
\pagecolor{light-yellow}
\normalsize

\vspace*{-1.5in}\parbox{10in}{
\parbox[c]{4.75in}{
\begin{center}
{\large Caveat\vspace*{1ex}}
\end{center}

These findings are valid provided
the effects of {\sl resonant inclination--pumping}
are modest.\\

This is clearly important for grains
with radii $R\gtrsim100\ \mu$m.\\

However most of the zodiacal light is reflected
by grains with $R\lesssim100\ \mu$m.\\

}
\hfil\vspace*{-2in}\parbox{5.0in}{
\begin{figure}[t]
\psfig{figure=figs/inc_grogan.eps,height=5.0in,width=5.0in}
\end{figure}
\begin{center}
{\tiny Figure from Grogan {\it et al} (2001)}
\end{center}
}
}

%%%%%%%%%%%%%%%%%%%%%%%%%%%%%%%%%%%%%%%%%%%%%%%%%%%%%%%%%%%%%%%%%
% 21. Asteroid Belt						%
%%%%%%%%%%%%%%%%%%%%%%%%%%%%%%%%%%%%%%%%%%%%%%%%%%%%%%%%%%%%%%%%%
\newpage
\pagecolor{light-yellow}
\normalsize

\vspace*{-1.3in}\begin{center}
{\large Extrapolate these Findings to the Asteroid Belt\vspace*{2ex}}
\end{center}
\noindent The total dust cross section is
\begin{displaymath}
\Sigma=\int\sigma(r,\beta)dV
\end{displaymath}
so $\Sigma_{total}(\mbox{2 AU})=1.6\times10^{10}$ km$^2\simeq
50\times\Sigma(\mbox{terrestrial planets})$.\\

\noindent Also, $\Sigma_{low}(\mbox{3.3 AU})=6.8\times10^{9}$ km$^2$ so
\begin{displaymath}
M_{low}(\mbox{3.3 AU})\sim\rho R_c\Sigma_{low}\sim
2\times10^{18}\left(\frac{\rho}{\mbox{2.5 gm/cm$^3$}}\right)
\left(\frac{R_c}{100\ \mu\mbox{m}}\right)\mbox{ gm}
\end{displaymath}
which is the mass equivalent of a $D\sim12$ km asteroid.

%%%%%%%%%%%%%%%%%%%%%%%%%%%%%%%%%%%%%%%%%%%%%%%%%%%%%%%%%%%%%%%%%
% 22. Oort Cloud						%
%%%%%%%%%%%%%%%%%%%%%%%%%%%%%%%%%%%%%%%%%%%%%%%%%%%%%%%%%%%%%%%%%
\newpage
\pagecolor{light-yellow}
\normalsize

\vspace*{-1.0in}\begin{center}
{\large Boldly Extrapolate out to the Oort Cloud\vspace*{2ex}}
\end{center}
\noindent Oort Cloud comets travel out to $a\sim10^4$ AU, and
the extrapolated mass of their dust is of order
\begin{displaymath}
M_{iso}\sim10^{19}\left(\frac{\rho}{\mbox{1 gm/cm$^3$}}\right)
\left(\frac{R_c}{1\ \mu\mbox{m}}\right)
\left(\frac{a}{10^4\mbox{ AU}}\right)\mbox{ gm}
\end{displaymath}
(albeit {\sl very} uncertain),
which has a mass equivalent to a $D\sim30$ km comet.


%%%%%%%%%%%%%%%%%%%%%%%%%%%%%%%%%%%%%%%%%%%%%%%%%%%%%%%%%%%%%%%%%
% 23. Stellar Dust Tails					%
%%%%%%%%%%%%%%%%%%%%%%%%%%%%%%%%%%%%%%%%%%%%%%%%%%%%%%%%%%%%%%%%%
\newpage
\pagecolor{light-yellow}
\normalsize

\parbox[t]{10in}{
\vspace*{-0in}\parbox{4.75in}{
\begin{center}
{\large Stellar Dust Tails?\vspace*{2ex}}
\end{center}

This Oort Cloud dust fills a volume $a\sim10^4$ AU across.\\

This dust is ultimately stripped from the Sun by the local
flow of interstellar gas and dust, forming a
{\sl stellar dust tail}.\vspace*{2in}
}
\hfil\vspace*{-2in}\parbox[h]{5.0in}{
\vspace*{-1in}\begin{figure}
\psfig{figure=figs/DShalebopp1.ps,height=5.0in,width=5.0in}
\end{figure}
\begin{center}
{\tiny comet Hale Bopp, courtesy Dave Schleicher}
\end{center}
}
}

%%%%%%%%%%%%%%%%%%%%%%%%%%%%%%%%%%%%%%%%%%%%%%%%%%%%%%%%%%%%%%%%%
% 24. Conclusions						%
%%%%%%%%%%%%%%%%%%%%%%%%%%%%%%%%%%%%%%%%%%%%%%%%%%%%%%%%%%%%%%%%%
%\newpage
\pagecolor{light-yellow}
\normalsize

\hspace*{-0.5in}\parbox{10.0in}{
\vspace*{-1in}\begin{center}
{\large Conclusions\vspace*{1ex}}
\end{center}

\begin{itemize}
\item at most 45\% of the dust cross section in the ecliptic
at $r=1$ AU is due to dust in asteroid--like orbits
(but this estimate also includes dust from JFCs).
The mass of this dust is equivalent to a \mbox{$D\sim12$ km} asteroid.

\item at least 90\% of the dust interior to a $r=1$ AU sphere
is in comet--like orbits

\item these findings are valid if inclination--pumping
is modest (which is likely).

\begin{itemize}

\item however the preferred solution would be to simultaneously
fit a {\sl dynamic} (rather than a {\sl static}) dust model
to the IRAS \& Clementine observations.

\end{itemize}

\item also, the Sun lies at the center of a $a\sim10^4$ AU--wide 
cloud of dust generated by OCCs. This dust is steadily removed
by interstellar gas \& dust flow, possibly forming a stellar dust
tail.

\end{itemize}

}

\end{document}


